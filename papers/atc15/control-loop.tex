\section{Control Loop}

The dataplanes are controlled by the control plane.
The control plane is a normal system daemon which has a centralized view of various exposed performance metrics of each dataplane.
Based on these metrics and a preconfigured control algorithm, it decides on the resource allocation for each dataplane.
The objective of the control plane is to assign the minimal amount of resources to each dataplane without reducing its achieved performance or compromising its expected SLA.
The SLA is expressed in terms of the tail latency.
The dataplane provides a set of metrics which are proportional to the latency of the hosted network application.
The network application latency is an end-to-end metric and as such it cannot be captured directly by the underlying operating system without prior knowledge of the application itself or its network protocol.
Since such coupling is not desired, the tail latency is calculated based on a set of metrics which can be reliably measured at the operating system level.

\subsection{Energy-optimal allocation of resources}

In this section, we describe the designed policy for achieving energy-optimal allocation of resources to dataplanes.
The initial state of the resource allocation is to dedicate a single core to the dataplane and set the CPU package at the lowest possible clock frequency.
The general strategy is simple and adheres to following pattern: to cope with increasing load, the control plane first increases the number of allocated cores, and when this number reaches a maximum value, the control plane gradually increases the clock frequency of the CPU package.
This strategy provides the closest achievable behavior to an ideal CPU whose energy consumption is linearly proportional to its utilization.
