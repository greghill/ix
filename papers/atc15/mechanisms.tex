\section{Mechanisms}

The control plane uses two distinct mechanisms to control resource usage.
It can alter the number of assigned CPU cores to each dataplane.
It can modify the CPU frequency clock.

IX boots with the configured maximum number of cores.
Each core spins waiting for network activity or commands from the control plane.
Upon the reception of the respective control plane command, IX executes a blocking system call from the retiring core.
That way, Linux regains control of the core and can either schedule other processes or decide to let the core idle, depending on Linux scheduler's decisions.
The IX thread blocks on a named pipe, which can be signaled by the control plane in order to wake up the IX thread which will recover ownership of the respective core and continue its processing.
\george{reference to exokernel revokation protocol}

Additionally, the control plane is able to control the operating clock frequency of the CPU.
Linux exposes the DVFS mechanism available in modern processors through its \texttt{/sys} filesystem.
The control plane writes to specially designated files in that filesystem in order to set the operating frequency of the CPU.
Consequently, the mode of operation of this mechanism is transparent to the execution of the dataplane.
There is no need to communicate the increase or decrease of processor clock frequency to the dataplane.
