

\begin{abstract}

  The conventional wisdom is that aggressive networking requirements,
  such as high packet rates for small messages and microsecond-scale
  tail latency, are best addressed outside the kernel, in a user-level
  networking stack.  We present \ix, a {\it dataplane operating
    system} that provides high I/O performance, while maintaining the
  key advantage of strong protection offered by existing kernels.
  \ix uses
  hardware virtualization to separate management and scheduling
  functions of the kernel (control plane) from network processing
  (dataplane). The dataplane architecture builds upon a native,
  zero-copy API and optimizes for both bandwidth and latency by
  dedicating hardware threads and networking queues to dataplane
  instances, processing bounded batches of packets to completion, and
  by eliminating coherence traffic and multi-core synchronization. We
  demonstrate that \ix outperforms Linux and state-of-the-art,
  user-space network stacks significantly in both throughput and
  end-to-end latency. Moreover, \ix improves by up to
  4.7x the throughput of widely deployed, key-value store
  at a given latency bound.


\end{abstract}
