

\begin{abstract}

  Web-scale applications are placing aggressive demands on the TCP/IP
  stack implementation in modern systems, such as high packet rates
  for small messages, microsecond-scale tail latency, support for
  hundreds of thousands of connections, and elastic use of system
  resources. The conventional wisdom is that such requirements are
  best addressed outside the operating system, in a user-level
  implementation of the networking stack.

  We present \ix, a {\it dataplane operating system} specifically
  designed for web-scale applications. \ix brings lessons from high
  performance middleboxes such as firewalls and load-balancers to the
  design of general systems and applications.  It separates management
  and scheduling functions of the kernel (control plane) from network
  processing (dataplane).  \ix optimizes for both bandwidth and
  latency by dedicating hardware threads and networking queues to
  dataplane instances, processing bounded batches of packets to
  completion, and by eliminating coherence traffic and multi-core
  synchronization. \ix provides a native, zero-copy API that
  explicitly exposes flow control to applications. However, \ix uses
  hardware virtualization to offer the same protection model as
  commodity operating systems.


  We demonstrate that \ix outperforms Linux and mTCP, a
  state-of-the-art user-space networking stack, by up to 14x and 2.5x
  respectively for throughput. The unloaded uni-directional latency for two \ix
  servers is 6.9\microsecond, while Linux and mTCP lead to latencies
  of 21\microsecond and 95\microsecond respectively. Moreover, \ix
  efficiently supports a hundred thousand connections and improves the
  throughput of a massively deployed, key-value store by up to 2.7x.

\end{abstract}
