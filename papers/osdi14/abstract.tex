

\begin{abstract}

Web-scale applications are redefining the requirements of system
software, and in particular in the way system software implements the
TCP/IP networking stack.  The classic requirements to provide moderate
connection scalability, but high streaming performance with minimized
CPU occupancy is being replaced by new requirements that include
dealing with very high packet rates, enabling latency-sensitive
applications to respond within microseconds, both in the average and
the tail latency case, the ability to support hundreds of thousands of
connections, possibly with high churn, and elastically manage system
resources.  The conventional wisdom is that such requirements are best addressed
outside of commodity operating system, either at userlevel or offloaded into
intelligent hardware.  

We present IX, a dataplane operating system specifically designed for
web-scale, event-driven applications.  IX addresses these new
application requirements through a novel architecture that uses
virtualization hardware to separate control plane from the dataplanes,
and to offer protection to the dataplane.  The IX kernel exposes a
native, zero-copy API that safely but explicitly exposes flow control
to untrusted software.  IX is designed to scale on multicore hardware
through a coherency-free, run-to-completion execution model.

In userspace, a compatiblility library exposes an API similar to
libevent to legacy applications.  Our evaluation using microbenchmarks
on that stack show that IX outperforms Linux on nearly all benchmarks
by one order of magnitude, and a state-of-the-art user-space
networking stack by a factor of XXX.  Our evaluation using real-world
applications simmilar gains, and that the porting effort is minimal.


\end{abstract}
