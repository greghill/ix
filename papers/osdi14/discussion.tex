
\section{Discussion}
\label{sec:disc}

We now discuss a number of lessons that were learned in the design and
implementation of \ix, some of its current limitations, and
opportunities for future work.

\myparagraph{What makes \ix faster?:} The results of \S\ref{sec:eval}
show that a networking stack can be implemented in an OS kernel,
provide protection and security to applications, and simultaneously
deliver wire-rate performance for most benchmarks.  The tight coupling
of the dataplane architecture, using only a minimal amount of batching
to amortize transition costs, causes application logic to be scheduled
at the right time, which is essential for latency-sensitive workloads.
The benefits of \ix can therefore exceed the time spent in the kernel \george{i don't understand that}
in a conventional Linux-based deployment.  The lack of intermediate
buffering pushes all abstractions outside of the kernel as in an
Exokernel~\cite{DBLP:conf/sosp/EnglerKO95}, allowing for efficient,
targeted implementations by applications.  In particular, the
zero-copy architecture helps even when the application libraries add a
level of copying, e.g., for \texttt{libevent} backward compatibility:
the copy occurs temporally much closer to its actual use, thereby
increasing cache locality.  Finally, we took great care in the
implementation of \ix to ensure that it could operate in a
coherency-free manner through the dataplane, with the guiding
principle to eliminate any atomic instructions from common case
execution.

% \myparagraph{Selecting \ix's TCP stack:} 
% The \ix kernel includes a complete network protocol stack with
% RFC-compliant support for TCP, UDP, ARP, ICMP and LACP.  It however
% explicitly lacks a complete IP-layer and is therefore only a capable
% networking endpoint, but not of handling routing, bridging or
% filtering functions.  It also totally lacks a socket layer and its
% associated API and semantics.  For our implementation of TCP, we chose
% to lwIP~\cite{dunkels2001design} as a starting point for our code base because of its
% modularity and is maturity as a compliant, feature-rich networking
% stack.  lwIP was however designed with memory-efficiency in mind,
% primarily for embedded environments.  We radically changed the data
% structures that organize protocol control blocks for scalability, as
% well as for finer-grain, precise timer management.  However, we did
% not (yet) attempt to optimize the code paths for performance, and
% consequently believe that our results have room for improvement.



\myparagraph{Challenges in API design:} \adam{TODO}


\myparagraph{Limitations of the prototype:} The \ix prototype is
architected to support any stateless offload NIC with multiple queues
and support for flow-consistent hashing in hardware (e.g., RSS).
However, we've only ported the Intel XXX driver, using
DPDK~\cite{intel:dpdk} as a starting code base.  We avoided adding to
\ix a Linux driver compatibility, explicitly trading off higher
portability costs for higher performance.  Our current implementation
operates without an IOMMU or VT-d; instead, the Dune environment maps
descriptor rings directly into \ix memory, and furthermore provides
\ix with the list of physical addresses in use, as \george{a?} form of
paravirtualization~\cite{DBLP:conf/sosp/BarhamDFHHHN03}.  Although
this implementation choice puts some level of trust into \ix, the
applications remain securely isolated.  Finally, \ix is designed and
implemented to support the dynamic addition and removal of elastic
threads, and the associated run-time rebalancing of hardware queues;
we've however only tested static configurations to date. 

\myparagraph{Limitations of RSS:} We realized that using RSS
makes the design of outgoing connections particularly complex.
Furthermore, current multi-queue NICs only support RSS groups of 16
queues, which is inadequate given current multi-core trends.  Despite
its flow scalability limitations, the use of Intel's Flow
Director~\cite{intel:82599} to provide connection
affinity~\cite{DBLP:conf/eurosys/PesterevSZM12} might provide an
interesting addition or alternative to our existing implementation.


\myparagraph{Opportunities for future work:} This paper focuses on the
OS aspects of \ix, with an original focus on event-oriented
applications such as memcached, application, and web servers.
However, the \ix design is not limited to such applications.  For
example, \edb{MAKE SOMETHING UP - JVM}

Despite its critical importance to our original evaluation, our choice
of a TCP stack was driven by considerations of interoperability and
software modularity, rather than by its networking behavior (which we
had to enhance substantially to meet the challenges of current
datacenter networking speeds).  Looking ahead, we envision that an
\ix-style model may provide a higher performance alternative to the
current OS-based and hypervisor-based implementations of virtual
switches~\cite{openvswitch} and IP-level features (e.g., iptables).
In particular, the security model of \ix, which enforces modularity
without introducing noticeable overheads, appears like a great match
to meet the protection, multi-tenancy, and scalability requirements of
network function virtualization~\cite{etsi:NFV}.  Also, we expect that
DCTCP~\cite{DBLP:conf/sigcomm/AlizadehGMPPPSS10} will be synergistic
with the \microsecond-level latencies and reduced buffering
characteristics of \ix deployments.  DCTCP requires
ECN~\cite{ramakrishnan2001addition} in the network; it could easily be
added at the end-host in \ix to detect congestion at the NIC edge.


\myparagraph{DANGLING:}


\todo Lessons learned: importance of hardware knobs and micro-architectural effects.


\todo Use of dataplane vs. virtual machines. 

