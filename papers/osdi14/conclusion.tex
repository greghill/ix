
\section{Conclusion}

We have described \ix, a dataplane operating system that leverages
hardware virtualization to separate the control plane, the \ix
dataplane instances that implement in-kernel network processing, and
the event-driven applications running on top of it.  \ix provides a
native, zero-copy API that explicitly exposes flow control to
applications. The \ix dataplane optimizes for both bandwidth and
latency by processing bounded batches of packets to completion and by
eliminating coherence traffic and synchronization on multi-core
systems. We show that \ix outperforms Linux 3.11 by 14x and mTCP by up
to 2.5 in terms of throughput. It also outperforms Linux and mTCP in
terms of latency by 3x and 13.7x respectively. \ix can support
efficiently a hundred thousand connections. Finally, we show that
porting \texttt{memcached} to \ix removes kernel bottlenecks and
improves throughput by up to 2.7x.


% The system is designed to ensure high packet rates, to offer
% micro-second latency response times to clients including in high
% occupancy situations, to scale to a large number of TCP connections,
% and to allow for a coarse-grain, yet elastic allocation of hardware
% resources.
