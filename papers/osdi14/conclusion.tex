
\section{Conclusion}

We have presented \ix, a dataplane operating system for event-driven,
web-scale applications.  Although event-driven applications have
become the norm in web-scale applications, they typically run on
commodity operating system and networking stacks designed and
optimized for the thread-oriented applications.  In contrast, our
architecture leverages hardware virtualization to separate the control
plane, the \ix kernel instances, and the event-driven applications
running on top of \ix.  The system is designed to ensure high packet
rates, to offer micro-second latency response times to clients
including in high occupancy situations, to scale to a large number of 
TCP connections, and to allow for a coarse-grain, yet
elastic allocation of hardware resources.

Our evaluation on stressful micro-benchmarks shows that \ix can
deliver wire-rate performance in most situations and typically
outperforms Linux 3.11 by an order of magnitude and mTCP, a
state-of-the-art user-level TCP stack, by a factor of three.  Our
latency-sensitive benchmarks indicate that it can deliver one-way
messages in \twiddle 7\microsecond.  Our evaluation with memcached, a
real-world key-value store shows that \ix can deliver XXXx queries per
second -- corresponding to XXX\% of the maximal wire rate at 4x10GbE -- with
latencies measured at the 99th percentile as low as XXX when operating
at 80\% of capacity. 


