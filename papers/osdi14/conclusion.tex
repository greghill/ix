
\section{Conclusion}

We described \ix, a dataplane operating system that leverages hardware
virtualization to separate and isolate the Linux control plane, the
\ix dataplane instances that implement in-kernel network processing,
and the network-bound applications running on top of it.  The \ix
dataplane provides a native, zero-copy API that explicitly exposes
flow control to applications. The dataplane architecture optimizes for
both bandwidth and latency by processing bounded batches of packets to
completion and by eliminating synchronization on multi-core
servers. On microbenchmarks, \ix noticeably outperforms both Linux and
mTCP in terms of both latency and throughput, scales to hundreds of
active concurrent connections, and can saturate 4x10GbE configurations
using a single processor socket.  Finally, we show that porting
\texttt{memcached} to \ix removes kernel bottlenecks and improves
throughput by up to \data{4.7}{4.1}$\times$.


