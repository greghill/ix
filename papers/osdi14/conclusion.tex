
\section{Conclusion}

We have described \ix, a dataplane operating system that leverages
hardware virtualization to separate the control plane, the \ix
dataplane instances that implement in-kernel network processing, and
the event-driven applications running on top of it.  \ix provides a
native, zero-copy API that explicitly exposes flow control to
applications. The \ix dataplane optimizes for both bandwidth and
latency by processing bounded batches of packets to completion and by
eliminating coherence traffic and synchronization on multi-core
systems. We show that \ix outperforms Linux \data{3.11}{3.16} by \data{14x}{XXX} and mTCP by up
to \data{2.5x}{XXXx} in terms of throughput. It also outperforms both Linux and mTCP in
terms of latency by \data{3x}{4x}. \ix can support
efficiently a hundred thousand connections. Finally, we show that
porting \texttt{memcached} to \ix removes kernel bottlenecks and
improves throughput by up to \data{2.7x}{4.5x}.


