
\section{Conclusion}

We have presented \ix, a dataplane operating system for event-driven,
web-scale applications.  Our architecture leverages hardware
virtualization to separate the control plane, the \ix dataplane
instances, and the event-driven applications running on top of it.
\ix optimizes for both bandwidth and latency by processing bounded
batches of packets to completion and by eliminating coherence traffic
and synchronization on multi-core systems. We show that \ix
outperforms Linux 3.11 by an order of magnitude and mTCP by up to 2.5
in terms of throughput. It also outperforms Linux and mTCP in terms of
latency by 3x and 15x respectively. \ix can support efficiently a
hundred thousand connections. Finally, we show that porting
\texttt{memcached} to \ix removes kernel bottlenecks and improves
throughput by up to 2.7x.

% A real-world key-value store,
% using two realistic synthetic benchmarks, shows that \ix can deliver
% more \twiddle  1.5M while meeting a service level aggreement of $1ms$ at the
% 95th percentile, an improvement of 1.8 and 2.7 over Linux, respectively.
% When running on \ix, memcached is no longer
% kernel-limited and its own application-level bottlenecks now limit its
% scalabilty.



% The system is designed to ensure high packet rates, to offer
% micro-second latency response times to clients including in high
% occupancy situations, to scale to a large number of TCP connections,
% and to allow for a coarse-grain, yet elastic allocation of hardware
% resources.
