
\section{Conclusion}

We have presented \ix, a dataplane operating system for event-driven,
web-scale applications.  Although event-driven applications have
become the norm in web-scale applications, they typically run on
commodity operating system and networking stacks designed and
optimized for the thread-oriented applications.  In contrast, our
architecture leverages hardware virtualization to separate the control
plane, the \ix kernel instances, and the event-driven applications
running on top of \ix.  The system is designed to ensure high packet
rates, to offer micro-second latency response times to clients
including in high occupancy situations, to scale to a large number of 
TCP connections, and to allow for a coarse-grain, yet
elastic allocation of hardware resources.

Our evaluation on stressful micro-benchmarks shows that \ix can
deliver wire-rate performance in most situations and typically
outperforms Linux 3.11 by an order of magnitude and mTCP, a
state-of-the-art user-level TCP stack, by a factor of two or more. Our
latency-sensitive benchmarks indicate that it can deliver one-way
messages in \twiddle 6\microsecond, or 15\microsecond less than Linux.
\edb{NEW:}Our evaluation with memcached, a real-world key-value store,
using two realistic synthetic benchmarks, shows that \ix can deliver
more \twiddle  1.5M while meeting a service level aggreement of $1ms$ at the
95th percentile, an improvement of 1.8 and 2.7 over Linux, respectively.
When running on \ix, memcached is no longer
kernel-limited and its own application-level bottlenecks now limit its
scalabilty.


