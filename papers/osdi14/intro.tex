
\section{Introduction}
\label{sec:intro}


Web-scale applications such as search, social networking, and
e-commerce platforms, are redefining the requirements on system
software. A single application can consist of hundreds of software
services, deployed on thousands of servers. For example, Amazon
accesses 150 distinct services to render each page requested by
users~\cite{DBLP:conf/sosp/DeCandiaHJKLPSVV07}. Such applications need
TCP/IP networking stacks that go well beyond the classical
requirements of high streaming performance and moderate connection
scalability. The new requirements include high packet rates for short
messages~\cite{Atikoglu:2012:WAL}, microsecond-level responses to
remote requests with tight tail latency
guarantees~\cite{DBLP:journals/cacm/DeanB13}, and support for hundreds
of thousands of connections with high connection
churn~\cite{DBLP:conf/nsdi/NishtalaFGKLLMPPSSTV13}. It is also desirable to be elastic
in resource usage, allowing other applications to use any available
resources in a shared
cluster~\cite{DBLP:journals/computer/BarrosoH07,DBLP:conf/eurosys/LeverichK14}.

The conventional wisdom is that there is a fundamental mismatch
between the requirements of web-scale workloads and existing
implementations of TCP/IP in commodity operating
systems. Consequently, many proposed solutions bypass the OS and
implement the networking stack in
user-space~\cite{jeong2014mtcp,DBLP:conf/cloud/KapoorPTVV12,openonload,DBLP:conf/sigcomm/MarinosWH14,DBLP:conf/sigcomm/ThekkathNML93}. Some
systems go one step further by also replacing TCP/IP with RDMA in
order to offload protocol processing to Infiniband
adapters~\cite{DBLP:conf/sosp/OngaroRSOR11,DBLP:conf/icpp/JoseSLZHWIOWSP11,mitchell:rdma,dragojevic14farm}. While
kernel bypass eliminates context switch overheads, on its own it does
not eliminate the difficult tradeoffs between high packet rates and
low latency or the challenges of managing large connections
counts. Moreover, user-level networking suffers from lack of
protection. Application bugs and crashes can corrupt the networking
stack and impact other workloads.


\dm{Again, this makes it sound narrow.  Phrase in a more fundamental
  way.  Point is that currently people must achieve a delicate $n$-way
  balance between throughput, latency, protection/robustness,
  complexity, number of machines/power consumption, etc.  \ix shows it
  doesn't have to be this way; we can have our cake and eat it, too,
  if we architect better systems around improved APIs.} 

We propose \ix, an operating system designed specifically to meet the
requirements of event-driven, web-scale applications.  Its
architecture builds upon the lessons from high performance
middleboxes, such as firewalls, load-balancers, and software
routers~\cite{DBLP:journals/tocs/KohlerMCJK00,DBLP:conf/sosp/DobrescuEACFIKMR09}. \ix
separates the control plane, which is responsible for basic kernel
functionality such as provisioning and scheduling, from the dataplanes
which run the networking stack and application logic. However, \ix
uses hardware
virtualization~\cite{DBLP:journals/computer/UhligNRSMABKLS05} to
isolate the dataplanes from the applications they run and offer the
same protection model as commodity operating systems. In our
implementation, the control plane runs on a full Linux kernel and the
dataplanes use Dune to run \ix as protected, library-based operating
systems on dedicated hardware
threads~\cite{DBLP:conf/osdi/BelayBMTMK12}.


\ix provides a native, zero-copy API that explicitly exposes network
flow-control to applications.  This API enables dataplanes to optimize
for both bandwidth and latency by processing a bounded batch of
packets to completion.  Each dataplane executes all the pipeline
stages for TCP/IP processing for the batch in kernel mode, followed by
the associated application processing in user mode. This approach
amortizes API overheads and improves both instruction and data
locality.  \ix is also optimized for synchronization and coherence
free execution on multi-core systems. It uses multi-queue network
adapters (NICs) and receive-side scaling (RSS)~\cite{url:rss} to
implement flow-consistent hashing of incoming traffic to distinct
hardware queues. Each dataplane instance exclusively controls a set of
these queues and runs the networking stack and a single application,
such as a web server or a distributed caching server. Moreover, the \ix
API and companion user-level library meet the commutativity
rule~\cite{DBLP:conf/sosp/ClementsKZMK13}, eliminating multi-core
synchronization in the common case operation. The \ix user-level
library includes an interface nearly identical to the popular
\texttt{libevent} library~\cite{provos2003libevent}, providing
compatibility between \ix and a wide range of existing applications.

\dm{The above paragraph does a reasonable job of describing what's
  new.  But it might be worth another paragraph addressing why now
  Basically there's a fundamental question of granularity here.  E.g.,
  page sizes have been 4--8K since machines had less than a Megabyte
  of DRAM\@.  Schedulers have been designed under the premise of more
  applications than CPUs.  Network APIs were designed when packet
  inter-arrival times were many times the system call/interrupt
  latency.}

\dm{Another point to add is that the breadth of OS APIs has made it
  virtually impossible to deploy clean-slate operating systems,
  despite possibly huge performance gains from radically different IO
  architectures.  Fortunately, the performance-critical IO functions
  are a small subset of the garbage can of system calls required for
  setup, initialization, and configuration.  Hence, a big contribution
  is showing how we can completely rearchitect the IO path while
  retaining a high degree of source code compatibility and remaining
  compatible with existing system configuration and management tool.}


We compare \ix against Linux \data{3.11.10}{3.16} and mTCP, a state-of-the-art
user-level TCP stack~\cite{jeong2014mtcp}.  \ix outperforms Linux and
mTCP by up to \data{14x}{XXX} and \data{2.5x}{XXX} respectively for throughput. \ix even
scales to 4x10GbE configuration using a single multi-core socket.  The
unloaded uni-directional latency for two IX servers is
\data{6.9}{XXX}\microsecond, while Linux and mTCP lead to latencies of
\data{21}{XXX}\microsecond and \data{95}{95}\microsecond respectively.
Our evaluation with memcached, a massively deployed key-value store,
shows that \ix improves upon Linux by up to \data{2.7}{XXX}x in terms of
throughput at a given 95th percentile latency bound, as it can reduce
kernel mode processing from $>80\%$ with Linux to $<33\%$ with \ix.

\ix demonstrates that, by revisiting the networking APIs and taking
advantage of modern NICs and multi-core chips, we can design systems
that achieve high throughput \underline{and} low latency
\underline{and} high connection counts \underline{and} robust
protection. It also shows that, by separating the small subset of
performance-critical I/O functions from the rest of the kernel, we can
design systems with radically different I/O architectures and achieve
large performance gains, while retaining compatibility with the huge
set of APIs and services provided by a modern OS like Linux.



The rest of the paper is 
organized as follows. \S \ref{sec:motivation} motivates the need
for a new OS architecture for web-scale applications. \S\ref{sec:design} and \S\ref{sec:impl} present the design principles and
implementation of \ix. \S\ref{sec:eval} presents the
quantitative evaluation.\S\ref{sec:disc} and \S\ref{sec:related} discuss future and related work.






