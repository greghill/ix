

\section{Related Work}
\label{sec:related}

We organize the discussion topically, while avoiding redundancy with
the commentary already exposed in \S\ref{sec:motivation:current}.


\myparagraph{Virtualization hardware:} Hardware support for
virtualization naturally separates control and execution functions,
e.g., to build type-2 hypervisors
~\cite{DBLP:journals/tocs/BugnionDRSW12,misc/kivity07kvm}, run virtual
appliances~\cite{DBLP:conf/lisa/SapuntzakisBCZCLR03}, or processes
with access to privileged instructions~\cite{belay2012dune}.
Arrakis~\cite{peter2013arrakis,arrakisTR13} uses virtualization
hardware, the Barrelfish
multikernel~\cite{DBLP:conf/sosp/BaumannBDHIPRSS09} for control, sharing
s a similar separation of the dataplane from the control
plane. \ix differs in that it uses Linux for
control, has isolation of dataplane and application through use of Dune \adam{Arrakis runs the application and the network stack in the
same protection domain, so this is a pretty large difference},
and has a different execution model built around adaptive batching and run to completion.


\myparagraph{Library operating systems:}
Exokernels proposed to implement system abstractions via library
operating systems linked in with
applications~\cite{DBLP:conf/sosp/EnglerKO95}, extending the
end-to-end principle~\cite{DBLP:journals/tocs/SaltzerRC84} to the
management of system resources.  Library operating systems typically
run as virtual machines ~\cite{DBLP:journals/tocs/BugnionDGR97},
e.g. to deploy cloud services
~\cite{DBLP:conf/asplos/MadhavapeddyMRSSGSHC13}. The \ix kernel limits
itself to the implementation of protocol stack, allowing applications
to implement their own resource management policies, e.g. via the
\texttt{libevent} compatibility layer.


\myparagraph{User-level networking stacks:} User-level network
stacks~\cite{jeong2014mtcp, marinos2013network, openonload} can
outperform kernel-based implementations through specialization and
elimination of redundant layers and abstractions, but trade-off
performance for a weaker security model.  The \ix dataplane
demonstrates that a specialized networking stack can offer performance
and cooperate with applications without having to weaken security and
isolation properties.
 

\myparagraph{Hardware and protocol specialization:} 
Applications can use a connection-less UDP-based protocol for
scalability~\cite{nishtala2013scaling}.  Latency-sensitive datacenter
applications can use specialized Infiniband adapters to expose RDMA
with $1-3$\microsecond latencies to
applications~\cite{DBLP:conf/sosp/OngaroRSOR11,Jose:2011:MDH,mitchell:rdma,dragojevic14farm}.
Specialized FGPAs can replace conventional servers for important
applications such as
memcached~\cite{DBLP:conf/fpga/ChalamalasettiLWARM13}.  \ix is
designed to allow TCP/IP to scale with architectural trends by
elminating kernel bottlenecks.


\myparagraph{Asynchronous and zero-copy interactions:} 
Systems with asynchronous, batched, or exception-less system calls
substantially reduce the overheads associated with frequent kernel
transitions and context switches~\cite{
  soares2010flexsc,han2012megapipe,rizzo2012netmap,jeong2014mtcp}.
Zero-copy reduce data movement overheads and simplifies resource
management~\cite{DBLP:journals/tocs/PaiDZ00}.  \ix's use of adaptive,
bounded batching is suitable for both low-latency and high-message
rate, and its safe, cooperative memory management model enables
zero-copy at the application level.


