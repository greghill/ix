
\section{Related Systems}

\paragraph*{Exokernel}   Exokernels~\cite{DBLP:conf/sosp/EnglerKO95} safely expose hardware to library operating systems. 


\paragraph*{Akkaris}.  
Akkaris~\cite{peter2013arrakis} is the closest system.  It is
apparently built on the barrelfish multikernel~\cite{DBLP:conf/sosp/BaumannBDHIPRSS09} and uses VT-x
like Dune~\cite{belay2012dune}.

\paragraph*{Megapipe} 
Megapipe~\cite{han2012megapipe} proposes a high-performance socket
alternative for TCP/IP based applications for scalable network
applications.  Our API is inspired by it. This is similar to
Netmap~\cite{rizzo2012netmap}, which also ping shared buffers, but is
focused on libpcap applications.

\paragraph*{Kernel-bypass approaches}  Solarflare Openonload~\cite{openonload}, ...

\paragraph*{Containers}  Linux containers, BSD jail, ...


\paragraph*{Application-specific kernels}  
Exokernels had a bunch of them.
SplashOS~\cite{DBLP:journals/tocs/BugnionDGR97} proposed this in the
context of hypervisors (Disco).  Libra~\cite{DBLP:conf/vee/AmmonsABSGKKRHW07} was a JVM that ran
directly on top of Xen.  It Relied on a separate Linux VM to provide
networking and storage (like Azul).  The
Unikernel~\cite{DBLP:conf/asplos/MadhavapeddyMRSSGSHC13} (ASPLOS 13)
lays out the case for BaremetalOS because it lays out the case for
single app operating systems.  The Unikernel ported the Ocaml runtime
to run on top of Xen hypervisor (like libra).  Actually makes a case
for reduced latency jitter, at least in terms of thread scheduling -
I/O performance basically sucks though because still not taking
advantage of HW (uses Xen PV-NIC).

\section{Techniques and Concepts}

\paragraph*{The C10M problem} As stated by Robert Graham(?)~\cite{theC10Mproblem}.


\paragraph*{Scheduler activations} 
If apps are able to yield cores, and if apps are able to
asynchronously take advantage of new cores, scheduler
activations~\cite{DBLP:journals/tocs/AndersonBLL92} seem like a
perfect mechanism to make those cores available to other processes. 
The idea never really made it into POSIX, but we can use here because the API is different. 

\paragraph*{Ballooning}
Ballooing was introduced for memory
management~\cite{DBLP:conf/osdi/Waldspurger02}.  The balloon keeps
track of the amount of relinquished resources.  CPU ballooning could
be used to (i) start a max core pool, and (ii) dynamically resize it
according to the SLA and the global load.

\paragraph*{Virtual memory performance}  
Direct segment mapping~\cite{DBLP:conf/isca/BasuGCHS13} proposes to
replace the traditional TLB with a dedicated direct segment.
Drepper~\cite{DBLP:journals/queue/Drepper08} measured additional cost
of running workloads in virtualized domains due to two-dimensional
page walking.

\paragraph*{Latency distribution}  
Dean and Barroso~\cite{DBLP:journals/cacm/DeanB13} describe the
problems associated with tail latency in web-scale services.
Dynamo~\cite{DBLP:conf/sosp/DeCandiaHJKLPSVV07} talked about the
importance to evaluate key-value stores based on their 99.9\% latency.


\paragraph*{Commutativity and scalability} 
The scalable commutativy rule~\cite{DBLP:conf/sosp/ClementsKZMK13}
states that only commutative APIs can be implemented in a truly
scalable manner; our API and our implementation should support that.
RadixVM~\cite{DBLP:conf/eurosys/ClementsKZ13} provides scalable
address spaces for multithreaded applications; we address this as well.

\paragraph*{Latency measurements}  

\section{Workloads}

\paragraph*{Memcached} 
Scaling memcached at Facebook~\cite{nishtala2013scaling} required
unnatural tricks such as relying on UDP and on in-network connection
proxies to compensate for the lack of scoket scalability of Linux.


\paragraph*{ngnx}




